\documentclass[a4paper,12pt]{article}

% Packages
\usepackage[utf8]{inputenc}
\usepackage[T1]{fontenc}
\usepackage{lmodern}
\usepackage{geometry}
\usepackage{graphicx}
\usepackage{amsmath}
\usepackage{hyperref}

% Page layout
\geometry{margin=1in}

% Document
\begin{document}

% Title Page
\title{Defaulted Credit Model Report}
\author{Dr. Luis Felipe Vidal Arellano}
\date{\today}
\maketitle

% Abstract
\begin{abstract}
This report provides an analysis of the defaulted credit model, including methodology, results, and conclusions.
\end{abstract}

\tableofcontents
\newpage

% Sections
\section{Descrição do problema}
A dívida ativa representa o conjunto de créditos tributários e não tributários definitivamente constituídos e não pagos pelos contribuintes, cuja inscrição formal permite ao ente público iniciar os procedimentos de cobrança administrativa ou judicial. Em um contexto municipal, essa dívida constitui não apenas um direito a ser exercido pela Fazenda Pública, mas também um ativo patrimonial de relevância estratégica para o financiamento de políticas públicas. No entanto, o volume crescente de créditos inscritos contrasta com a baixa taxa de recuperação efetiva observada, revelando uma desconexão entre a dimensão jurídica da dívida e sua viabilidade econômica.

Diversos fatores explicam essa baixa recuperabilidade. Grande parte da carteira é composta por créditos antigos, com débitos de pequeno valor e devedores que frequentemente são pessoas físicas de baixa renda ou empresas inativas. O processo de cobrança, sobretudo quando judicializado, tende a ser moroso e oneroso, gerando custos que muitas vezes superam os valores efetivamente recuperados. Além disso, a ausência de informações estruturadas sobre o perfil dos devedores, bem como a dificuldade de integração entre os sistemas administrativos, financeiros e judiciais, compromete a capacidade do gestor público de avaliar o risco de inadimplemento e de formular estratégias diferenciadas de cobrança.

Tradicionalmente, os entes públicos têm adotado critérios genéricos e pouco analíticos para a gestão da dívida ativa, baseando-se em valores nominais, ordem cronológica de inscrição ou fase processual para decidir sobre o ajuizamento de execuções fiscais ou a adoção de medidas de cobrança. Essa prática, embora administrativamente simples, ignora a profunda heterogeneidade da carteira e negligencia fatores que afetam diretamente a probabilidade de recuperação, como a situação econômica do devedor, a existência de garantias, a natureza do crédito e o tempo de inadimplência.

A ausência de modelos quantitativos para estimar a recuperabilidade dos créditos inscritos gera consequências práticas relevantes. Em primeiro lugar, impede que a administração pública dimensione corretamente o valor econômico de seus ativos, dificultando a avaliação de riscos fiscais e a elaboração de projeções de receita mais realistas. Em segundo lugar, compromete a eficiência na alocação de recursos humanos e financeiros dedicados à cobrança. Por fim, dificulta a utilização da dívida ativa como instrumento de política pública, seja por meio de programas de regularização fiscal, seja por operações de cessão ou securitização de créditos.

Nesse cenário, justifica-se o desenvolvimento de um modelo estatístico capaz de estimar, para cada crédito individualmente, a probabilidade de recuperação e o valor esperado a ser recuperado. A proposta parte do reconhecimento de que a dívida ativa, apesar de seu caráter jurídico indiscutível, possui natureza econômica incerta e heterogênea. Assim, a construção de um modelo preditivo permite que a gestão da cobrança se oriente por critérios de risco e retorno, aproximando-se das práticas utilizadas no setor privado para avaliação de carteiras de crédito.

O presente trabalho apresenta um modelo integrado de estimativa de recuperabilidade da dívida ativa, que utiliza técnicas de regressão logística, regressão linear, modelos de sobrevivência e simulações de Monte Carlo para produzir estimativas probabilísticas individualizadas de recuperação. Inicialmente desenvolvido com base em dados simulados, o modelo foi concebido para aplicação prática em contextos reais, sendo compatível com as bases de dados e a estrutura administrativa disponíveis em grande parte dos municípios brasileiros. 

\section{Geração dos Dados Históricos e Estrutura da Base de Simulação}

A construção do modelo de estimativa da recuperabilidade da dívida ativa partiu, em sua fase inicial, da elaboração de uma base de dados simulada. Essa estratégia metodológica foi adotada com o propósito de validar a estrutura conceitual e estatística do modelo em ambiente controlado, no qual as relações entre variáveis podem ser especificadas de forma precisa e testadas com rigor. A ausência de dados públicos suficientemente detalhados e estruturados, sobretudo no que diz respeito a atributos dos devedores e informações consolidadas sobre recuperação efetiva, tornou necessária a adoção de uma base sintética, concebida para refletir com verossimilhança os padrões observados nas administrações tributárias municipais.

A base simulada foi composta por dois mil registros, cada um representando um crédito individual inscrito em dívida ativa. Esses créditos foram divididos em dois grupos conforme o estágio da cobrança: um grupo composto por 1.200 créditos em fase amigável e outro por 800 créditos já ajuizados. A separação por estágio permite distinguir a dinâmica de recuperação nos dois momentos processuais, uma vez que a probabilidade de êxito e o comportamento dos devedores variam substancialmente conforme a antiguidade da dívida, a presença ou não de medidas constritivas e o grau de deterioração do crédito.

Para cada crédito, foram atribuídas variáveis representativas de seus atributos econômicos e jurídicos relevantes. A primeira delas é o escore de risco do devedor, denominado Credit_Score, simulado como uma variável contínua com distribuição normal. No grupo de cobrança amigável, essa variável foi gerada com média 450 e desvio padrão 50, enquanto no grupo judicial os mesmos parâmetros foram deslocados para uma média inferior (400), mantendo o mesmo desvio, refletindo a suposição de que os créditos ajuizados concentram devedores de maior risco.

Outra variável essencial é a idade da inadimplência, representada por Delinquency_Age, que corresponde ao tempo decorrido, em meses, entre o vencimento da obrigação e o momento de análise. Optou-se por uma distribuição uniforme nesse caso, variando entre 6 e 18 meses na fase amigável, e entre 12 e 24 meses na fase judicial, o que reproduz o atraso adicional necessário para o ajuizamento de execuções fiscais.

A variável LTV, ou razão entre o valor do crédito e o valor das garantias disponíveis (Loan-to-Value), foi construída com base em uma distribuição assimétrica, com predominância de valores próximos a 1 ou superiores, refletindo a realidade dos créditos públicos, que em sua maioria não contam com garantias reais específicas. Essa variável foi tratada como contínua e adimensional, podendo ser utilizada tanto como proxy da fragilidade do crédito quanto como instrumento para segmentação da carteira.

Com base nessas variáveis explicativas, foi construída uma variável latente de propensão à recuperação, por meio de uma função logística da forma:


 
Os parâmetros dessa função foram calibrados de modo a produzir taxas médias de recuperação de aproximadamente 40% para os créditos amigáveis e 15% para os judiciais, em consonância com as estimativas comumente observadas em estudos empíricos no Brasil. A realização de sorteios binários com base nessa probabilidade gerou a variável observada Recovered.

Nos casos em que o crédito foi considerado recuperado (i.e., quando Recovered = 1), foi atribuído um valor recuperado efetivo por meio de uma função proporcional ao valor original do crédito, ponderado por variáveis de risco. Embora o valor de face não tenha sido explicitado no modelo, admite-se implicitamente que ele se encontra normalizado e que o valor recuperado pode ser expresso como:

onde 

V é o valor do crédito (tomado como unidade de referência) e r é uma taxa de recuperação simulada a partir de distribuições condicionalmente parametrizadas para cada grupo.

Essa base simulada fornece o substrato necessário para a estimação dos modelos estatísticos utilizados nas etapas seguintes, permitindo a aplicação de regressões logísticas para a probabilidade de recuperação, de regressões lineares para variáveis latentes e da modelagem paramétrica da taxa de recuperação via distribuição Beta. Além disso, a simulação facilita a realização de testes de Monte Carlo e análises de sensibilidade, fundamentais para validar a coerência interna do modelo e verificar sua resposta frente a variações de parâmetros.

Embora se trate de uma base fictícia, a estrutura de variáveis adotada busca refletir, com razoável fidelidade, os dados que podem ser coletados nos sistemas administrativos dos entes públicos, o que será discutido na próxima seção. A criação desse ambiente simulado, portanto, não tem apenas finalidade ilustrativa, mas constitui uma etapa metodologicamente indispensável para o amadurecimento e a futura aplicação empírica do modelo em contextos institucionais reais.

\section{Results}
Present the findings of the analysis.

\section{Conclusion}
Summarize the key takeaways and potential future work.

% References
\newpage
\bibliographystyle{plain}
\bibliography{references}

\end{document}